\setlength{\absparsep}{18pt} % ajusta o espaçamento dos parágrafos do resumo
\begin{resumo}
O presente trabalho apresenta um sistema de IoT que visa auxiliar unidades de emergências em meio ao trânsito, comunicando aos veículos que estão na rota desta unidade que ela passará por lá e eles devem abrir caminho, além disso possibilitar o armazenamento de dados para uma futura análise sobre os veículos conectados a arquitetura desenvolvida, para que se obtenha assim indicadores sobre o trânsito. Para tal desenvolvimento utilizamos de ferramentas novas no mercado, de forma a buscar um novo modelo de arquitetura que venha a ser robusta, escalável e de fácil manipulação e integração com outros projetos depois de desenvolvida. Tais objetivos foram alcançados de forma que quando realizados testes de \textit{stress} e simulação a arquitetura se comportou de forma satisfatória apresentando respostas consideradas rápidas validando assim a arquitetura, conseguimos a agilidade durante o desenvolvimento através das ferramentas utilizadas, como é o caso do encapsulamento de serviços em Docker.

 \textbf{Palavras-chave}: IoT. Big Data. Clojure.
\end{resumo}