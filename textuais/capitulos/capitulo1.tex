\chapter{Internet das Coisas Aplicações em Mobilidade Urbana e Saúde}
\label{chap:cap1}
Internet das coisas, conhecido também como IoT, sigla que em inglês significa \textit{Internet Of Things}, originou-se através de Kevin Ashton que em 1999 realizou uma apresentação na empresa  Procter \& Gamble (P\&G), quando falava em se etiquetar eletronicamente os produtos da empresa através do uso de Identificador de Rádio Frequência (RFID), assunto que era recente na época. Desde então este paradigma tem sido muito discutido, principalmente no contexto atual, em que é possível notar um crescimento exponencial de tecnologias desenvolvidas neste sentido, como é mostrado na figura \ref{fig:graficoIot2011-2025} que apresenta o aumento no uso de IoT mundialmente, fazendo uma estimativa até o ano de 2025.\cite{historiaiot} 

\begin{figure}[htb]
\caption{\label{fig:graficoIot2011-2025}Gráfico de crescimento do IoT entre os anos de 2011 a 2025}
\begin{center}
\includegraphics[scale=0.75]{graficoIot2011-2025}
\end{center}
\legend{Fonte: IHS 2013}
\end{figure}

Tais dados se devem as consequências geradas pela emergência de tecnologias microeletrônicas, \textit{wireless} (\textit{Wi-fi}, \textit{Bluetooth} e \textit{ZigBee}), interfaces de comunicação móveis que se somaram as fixas já existentes e devido a formação de uma grande rede ubíqua capaz de conectar seres humanos com uma grande facilidade, possibilitando assim fornecer toda a base para a formação da IoT. \cite{santaella2013} 

\section{O que é IoT?}
\label{sec:oqueeiot}
No conceito de IoT um terceiro elemento foi inserido nas redes pervasivas que se possui hoje em dia, os objetos, sendo assim dentro da rede é possível se ter a comunicação entre humano-humano, humano-objeto e objeto-objeto, desta forma é possível ter humanos se comunicando normalmente como já acontecia anteriormente, humanos definindo comportamentos para os objetos e recebendo dados dos mesmos e objetos trocando informações entre si disponibilizando dados a humanos, dados estes, úteis para tomada de decisões ou até mesmo para facilitar atividades do dia a dia.\cite{santaella2013}

\begin{citacao}
Quando os objetos podem sentir o ambiente e se comunicar, eles se tornam ferramentas poderosas para entender coisas complexas e responder a elas com eficiência. Embora tais objetos inteligentes possam interagir com humanos, é mais provável que interajam ainda mais entre si automaticamente, sem intervenção humana atualizando-se com as tarefas do dia.\cite[p. 2]{presser2011}
\end{citacao}

Tais objetos podem ser considerados como tudo que está na rede e possui um endereçamento \textit{Internet Protocol} (IP), podendo interagir com outras interfaces endereçáveis dentro da mesma rede ou em outras através da internet, como mostra na figura \ref{fig:fluxogramaiot}.

\begin{figure}[htb]
\caption{\label{fig:fluxogramaiot} Fluxograma da IoT}
\begin{center}
\includegraphics[scale=0.75]{fluxogramaiot}
\end{center}
\legend{Fonte: IHS 2013} 
\end{figure}

Esses objetos podem ser um automóvel, uma geladeira, uma câmera, um sensor de temperatura, entre muitas outras interfaces, o que importa é que elas estão interligadas pela internet tomando ações de forma automática sem a intervenção humana. Pode-se citar o exemplo de um senhor que sofre de mal de Alzheimer e mora sozinho sendo que seus filhos não podem estar 24 horas por dia com ele, então os filhos decidem implantar sensores na casa do pai e pela vizinhança para que possam saber remotamente aonde ele está. Estes sensores estariam conectados a internet enviando dados para os filhos e emitindo alertas caso o pai saia de casa.\cite{presser2011}

Um outro exemplo de aplicação da IoT é apresentado na figura \ref{fig:exemploemergenciasiot}.

\begin{figure}[htb]
\caption{\label{fig:exemploemergenciasiot} Exemplo de aplicação da IoT}
\begin{center}
\includegraphics[scale=0.4]{exemploemergenciasiot}
\end{center}
\legend{Fonte: \citeauthor{presser2011}, \citeyear{presser2011}} 
\end{figure}

Para que as aplicações de IoT tenha este tipo de comportamento é necessário que se tenha uma infraestrutura para dar suporte a esses objetos, ela pode ser estruturada de diferentes formas utilizando diversas tecnologias, mas de modo geral para o funcionamento de um sistema de IoT é necessário que se tenha os objetos conectados na internet ou a uma rede local, que envie e receba dados da infraestrutura (banco de dados ou armazenamento na nuvem) e os aplicativos que tem a função de gerenciar o sistema acessam e enviam os dados se comunicando diretamente com a infraestrutura, como é mostrado na figura~\ref{fig:estruturaiot}

\begin{figure}[htb]
\caption{\label{fig:estruturaiot} Estrutura de um sistema de IoT}
\begin{center}
\includegraphics[scale=0.5]{estruturaiot}
\end{center}
\legend{Fonte: IHS, 2013} 
\end{figure}

\section{Smart Cities}
\label{sec:smartcities}
\lipsum[3]



%\autoref{chap:cap1}
% ---
%\section{Aliquam vestibulum fringilla lorem}
%\lipsum[2]
%\subsection{Subsessão cap 1}
%\lipsum[2]
%\chapter{Capitulo Segundo}
%\lipsum[2]