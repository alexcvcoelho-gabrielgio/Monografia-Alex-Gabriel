\chapter{Especificações e Arquitetura do Sistema}
\label{chap:arquitetura}
Durante todo o desenvolvimento do sistema, buscou-se não se utilizar dos modelos de arquitetura comumente empregados neste tipo de aplicação, que trabalham com grandes volumes de dados, um dos motivos vem a ser pela grande complexidade encontrada ao se trabalhar com módulos Apache como o Hadoop Common, Hadoop Distributed File System (HDFS), Hadoop YARN e Hadoop MapReduce, os quais dependendo de suas aplicações acabam por gerar dependências de outras ferramentas como Cassandra, Spark, ZooKeeper, entre outras que se mostram apesar de robustas, muito complexas e com uma curva de aprendizagem alta demais para serem usadas de imediato.

Outro motivo de grande valor para o desenvolvimento de uma nova arquitetura tem sido a empregabilidade de novas ferramentas utilizadas no lado \textit{backend}, por assim dizer, relacionado a organização dentro dos servidores, conceitos de desenvolvimento e bancos de dados que tem ganhado força dentro do mercado e se mostradas bem aplicáveis em projetos com grandes volumes de dados e alta disponibilidade, além de possuir uma baixa curva de aprendizagem, se comparadas com as tecnologias tradicionais.

\section{Docker}
\label{sec:docker}
Docker é uma ferramenta criada sob os conceitos de conteinerização de aplicações, mudando os processos realizados pelos profissionais de infraestrutura, em que, de forma tradicional era criava-se um servidor no qual se instalava todas as ferramentas exigidas pela aplicação para que ela funcionasse corretamente, despendendo de horas de trabalho para tal feito. Com o uso do Docker a aplicação está contida dentro de um contêiner, podendo se relacionar ou não com outros contêineres, desta forma se tem a imagem do contêiner na qual está contida todos os arquivos e configurações necessárias para que uma instância da aplicação seja colocada em funcionamento.

Esta ferramenta é comumente comparada com a criação de máquinas virtuais, pois assim como estas, em uma única máquina é possível se ter várias divisões dos processos.

\section{Tecnologias}
\label{sec:tecnologias}
Mostrar de forma específica cada tecnologia que será empregada

\subsection{Clojure}
Apresentar a linguagem de programação \textit{Clojure} e como ela é empregada no trabalho.

\subsection{Docker e Docker Swarm}
Apresentar o uso de containers \textit{Docker} para encapsular componentes do sistema e \textit{Docker Swarm} para balancear cargas entre containers.

\subsection{Redis}
Mostrar o uso de um \textit{cache} dentro do sistema, sua importância e melhora no desempenho do sistema.

\subsection{Kafka}
Apresentar seu uso para realizar funções em tempo real dentro do sistema, bem como seu funcionamento.

\subsection{MongoDB}
Apresentar as características do banco NoSQL seu uso e diferenças entre um banco SQL, bem como seus pontos positivos.

\subsection{Datomic}
Apresentar uma nova forma de armazenar dados, se utilizando de histórico, sempre mantendo o estado anterior dos dados.

