\chapter{Introdução}
%\addcontentsline{toc}{chapter}{Introdução}
A internet das coisas (IoT) vem sendo muito difundida, ganhando cada vez mais força no mercado atual, se tornando uma tendência das novas aplicações que se tem desenvolvido, ela se caracteriza por ser a comunicação entre objetos dentro de uma rede de dados, de forma a resolver problemas e trazer facilidades para os usuários desta tecnologia, no trânsito ela vem sendo implantada nos computadores de bordo dos carros, como pode ser visto em carros de diversas marcas, em que pelo painel do veículo podem ser agendadas revisões periódicas do veículo, bem como realizar troca de informações com a central de ajuda da concessionária.

O IoT ainda tem muito a ser explorado dentro do trânsito, pois hoje se tem carros que carregam consigo uma vasta gama de tecnologias com relação a sensores e dispositivos de segurança, mas ainda se tem pouco das tecnologias de IoT empregadas dentro de um carro, muito menos uma infraestrutura robusta para suportar veículos conectados a ela, mas é um ramo que está em crescimento e é questão de tempo para que estas tecnologias passem a surgir, muitos estudos e protótipos já se tem feito, como é o caso do carro autônomo, em que em algum momento ele passará a existir.

Tratando-se de trânsito, os congestionamentos tem se tornado um problema comum nas grandes cidades, dificultando assim a circulação de unidades de emergência que se deslocam para realizar o atendimento a vítimas, sendo que um minuto a mais ou a menos pode significar a vida de uma pessoa, esta situação é problemática e complexa de ser resolvida, pois mesmo que os motoristas abram caminho para a unidade de emergência isto demanda um certo tempo, sem contar que muitas vezes não é possível que se abra caminho para a passagem da mesma.

Tendo em vista esta problemática, o presente trabalho visa se utilizar de tecnologias de IoT para resolver este problema, criando uma arquitetura que seja robusta, escalável e distribuída o suficiente para suportar a grande quantidade de veículos que trafegam nas vias públicas e realizaram troca de dados com a infraestrutura através do conceito de \textit{Vehicle to Infrastructure} (V2I).


