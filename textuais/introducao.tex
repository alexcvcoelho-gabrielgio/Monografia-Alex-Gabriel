\chapter{Introdução}
%\addcontentsline{toc}{chapter}{Introdução}
A internet das coisas (IoT) vem sendo muito difundida, ganhando cada vez mais força no mercado atual, se tornando uma tendência das novas aplicações que se tem desenvolvido, ela se caracteriza por ser a comunicação entre objetos dentro de uma rede de dados, de forma a resolver problemas e trazer facilidades para os usuários desta tecnologia, no trânsito ela vem sendo implantada nos computadores de bordo dos carros, como pode ser visto em carros de diversas marcas, em que pelo painel do veículo podem ser agendadas revisões periódicas do veículo, bem como realizar troca de informações com a central de ajuda da concessionária.~\cite{chevrolet}

O IoT ainda tem muito a ser explorado dentro do trânsito, pois hoje se tem carros que carregam consigo uma vasta gama de tecnologias com relação a sensores e dispositivos de segurança, mas ainda se tem pouco das tecnologias de IoT empregadas dentro de um carro, muito menos uma infraestrutura robusta para suportar veículos conectados a ela, mas é um ramo que está em crescimento e é questão de tempo para que estas tecnologias passem a surgir, muitos estudos e protótipos já se tem feito, como é o caso do carro autônomo, em que em algum momento ele passará a existir.

Tratando-se de trânsito, os congestionamentos tem se tornado um problema comum nas grandes cidades, dificultando assim a circulação de unidades de emergência que se deslocam para realizar o atendimento a vítimas, sendo que um minuto a mais ou a menos pode significar a vida de uma pessoa, esta situação é problemática e complexa de ser resolvida, pois mesmo que os motoristas abram caminho para a unidade de emergência isto demanda um certo tempo, sem contar que muitas vezes não é possível que se abra caminho para a passagem da mesma.

Tendo em vista esta problemática, o presente trabalho visa se utilizar do conceito de IoT para resolver este problema, de forma que existindo uma interface embarcada com acesso a internet nos veículos, estes recebam alertas antecipadamente de que uma unidade de emergência passará por aquela rota e ele deve abrir caminho. Para isso a unidade de emergência deve comunicar ao sair atender a vítima, qual a rota percorrerá e a infraestrutura se encarregará de informar os veículos que estiverem presente na mesma rota, esta interface embarcada também poderá enviar informações relevantes sobre o veículo para  a infraestrutura, desta forma possibilitando diversas análises de dados.

Desta forma o trabalho visou a criação de uma arquitetura que seja robusta, escalável e distribuída o suficiente para suportar a grande quantidade de veículos que trafegam nas vias públicas e realizaram troca de dados com a infraestrutura através do \textit{Vehicle to Infrastructure} (V2I), um outro conceito dentro de IoT. Para isso buscou-se utilizar de novas ferramentas para o desenvolvimento desta arquitetura que também está relacionada com o \textit{Big Data} pois nela se tem um banco de dados exclusivo para este tipo de análise de dados, dentre elas Docker, Clojure, arquitetura CQRS, MongoDB e Datomic como bancos de dados, não se prendendo aos padrões tradicionais de \textit{Big Data}.

A escolha do tema a ser abordado se deu pela filosofia do IoT, que é proporcionar facilidades e melhor qualidade de vida aos seus usuários, desta forma utilizá-lo no trânsito para resolver um problema complexo como este faz todo sentido, além de se tratar de uma causa muito nobre que são as vidas das pessoas.

Tal trabalho se torna uma grande contribuição para a comunidade em geral, tanto a população que passará a ter um atendimento mais ágil no caso de emergências, salvando assim mais vidas que necessitem de uma unidade de emergência, seja uma ambulância ou bombeiro, quanto a comunidade \textit{Open Source} cuja a qual terá os códigos abertos para continuidade do projeto ou uso em outros tipos de projetos que possam se utilizar da arquitetura desenvolvida.

O trabalho está dividido em dois capítulos teóricos nos quais são apresentados todos os conceitos envolvidos no trabalho e juntamente com o funcionamento das tecnologias neles envolvidos, um capítulo explicando as ferramentas utilizadas, bem como a organização destas para o desenvolvimento da arquitetura, um capítulo de resultados onde é mostrado o que se conseguiu atingir através do uso da arquitetura, alguns casos de testes e a discussão destes resultados, e por fim a conclusão em que é mostrado os pontos relevantes sobre o trabalho.
