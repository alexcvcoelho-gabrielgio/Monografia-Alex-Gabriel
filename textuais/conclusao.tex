\chapter{Conclusão}
\label{chap:conclusao}
O desenvolvimento do presente sistema possibilitou a criação de uma nova arquitetura para um sistema de \textit{Big Data}, pois como apresentado, o sistema leva características dos 5Vs presente presente neste tipo de arquitetura, contendo um grande volume de dados, pois o sistema quando em operação gera uma massa de dados gigantesca vinda das milhares de conexões que se tem com todos os carros que estão no trânsito, as requisições são atendidas com uma alta velocidade, pois neste tipo de aplicação se torna inadmissível a demora na obtenção de uma resposta, e mesmo com o grande volume de requisições o sistema consegue manter a veracidade dos dados, tudo isso sem um alto consumo de recursos de \textit{hardware}, já que este pode não estar concentrado em um único ponto, mas sim distribuído entre várias máquinas. Sendo assim, a arquitetura e o sistema proposto realmente atendem os resultados esperados com relação a urgências no trânsito e prova que poderia ser utilizado em outros estudos de caso.

Outro ponto que se esperava com a arquitetura era a fácil utilização da mesma, diferente do que ocorre com as ferramentas tradicionais com relação a instalação e uso propriamente dito, a arquitetura especificada por possuir ferramentas dedicadas a instalação e gerenciamento das instâncias se apresenta mais simples e rápida de ser manipulada e através dos micro serviços que também podem ser chamados de \textit{Application Programming Interface} (API) quando analisado na visão da interface de usuário, a integração com outros sistemas os quais enviem ou consumam dados se torna fácil, mantendo a alta escalabilidade do sistema, tanto em relação a distribuição, como em relação aos dados armazenados.

Dentre as principais dificuldades durante a montagem da arquitetura especificada, está a dificuldade em se trabalhar com Clojure, apesar de todas as características e benefícios já mencionadas anteriormente, por ser uma linguagem recém lançada no mercado e não possuir uma comunidade ativa para a resolução de problemas encontrados com a linguagem, encontrar suporte e documentação para o desenvolvimento se tornou uma tarefa difícil, se fazendo necessário a análise puramente de \textit{logs}, que por muitas vezes não eram específicos da linguagem, mas sim da JVM, o que tornou mais difícil a obtenção de soluções, embora torne o aprendizado mais rico, este tipo de análise demanda maior tempo.

Em uma primeira fase do projeto ao invés de Kafka como fila de mensagens utilizou-se o Redis, apesar de se mostrar eficiente, limitava a questão de escalabilidade e distribuição do sistema, pois ele vem a ser uma fila simples sem replicação das filas como acontece no Kafka, desta forma ele foi substituído pelo Kafka que apesar de mais robusto também se apresentou mais complexo nas suas configurações, principalmente com relação ao encapsulamento em contêiner.

Como trabalho futuro para a arquitetura apresentada se tem o desenvolvimento da interface de usuário, que vem a ser o dispositivo embarcado presente nos carros, juntamente com a rede pela qual as informações trafegarão entre a infraestrutura e o veículo, para a escolha de ambas é necessário que se faça uma análise para a decisão de qual sistema de IoT deve ser utilizado, podendo se chegar ainda a conclusão de que a melhor forma seria o uso do celular do motorista conectado a rede 3G, pois a arquitetura permite comunicação com qualquer interface de usuário, provando mais uma vez a sua escalabilidade. 

Por apresentar uma causa nobre e escolha dos autores, todo o projeto está aberto em um repositório no GitHub (https://github.com/alexcvcoelho-gabrielgio) sob a licença GPL2, onde é possível encontrar todos os códigos desenvolvidos, configurações necessárias e \textit{links} de referência para as tecnologias, sendo assim, o trabalho fica aberto a possíveis colaboradores ajudarem a definir os próximos passos e direções a serem tomadas com o projeto.

